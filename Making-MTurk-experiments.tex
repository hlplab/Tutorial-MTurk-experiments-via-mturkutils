\documentclass{article}
\usepackage[margin=1in]{geometry}
\usepackage{xcolor}
\usepackage{adjustbox}
\usepackage{verbatim}
\definecolor{shadecolor}{rgb}{.9, .9, .9}
\usepackage{tcolorbox}
\usepackage[colorlinks = true,
            linkcolor = blue,
            urlcolor  = blue,
            citecolor = blue,
            anchorcolor = blue]{hyperref}
            
\usepackage{fvextra}
\DefineVerbatimEnvironment{codewrap}{Verbatim}{breaklines,commandchars=\\\{\}}

\newenvironment{code}%
   {\par\noindent\adjustbox{margin=1ex,bgcolor=shadecolor,margin=0ex \medskipamount}\bgroup\minipage\linewidth\codewrap}%
   {\endcodewrap\endminipage\egroup}

%\DeclareUrlCommand{\burl}{\def\UrlFont{\ttfamily\color{blue}}}

\newcommand{\burl}[3][blue]{\href{#1}{\color{#2}{#3}}}%

\title{Programming MTurk experiments via {\em JSEXP}}
\author{Florian Jaeger with help from Linda Liu, Zach Burchill, Wednesday Bushong, and Xin Xie}

\begin{document}

\maketitle

\tableofcontents 

\section{Overview}

Make sure you have completed the tutorials on {\em Setting up Mechanical Turk} and {\em Conducting Mechanical Turk experiments}. This tutorial walks you through how to program your own experiment using {\em \href{https://github.com/hlplab/JSEXP/}{JSEXP}}, which is a collection of Javascript programs originally written by Dave Kleinschmidt and then extended and modified by Zach Burchill, Wednesday Bushong, Esteban Buz, Florian Jaeger, Linda Liu, Xin Xie, and others.


\section{Disabling CORS errors when testing locally}
\href{https://medium.com/swlh/avoiding-cors-errors-on-localhost-in-2020-5a656ed8cefa}{Avoiding CORS Errors on Localhost (in 2020)}

To disable web security for Chrome, for example:

\begin{code}
open -n -a /Applications/Google\ Chrome.app/Contents/MacOS/Google\ Chrome --args --user-data-dir=”/tmp/chrome_dev_test” --disable-web-security
\end{code}

\section{Making your own experiment}

%current idea for workflow:
%develop script and test on laptop (localhost; student): select condition, list, etc. through URL parameters
%test on slate (florian upload:  test on students)
%sandbox from slate (florian)
% deploy from slate (florian)

\subsection{What needs to be in the HTML file?}

\begin{itemize}
    \item A form element that posts the resuls to MTurk when the experiment is done. 
    % ASK WEDNESDAY WHAT NEEDS TO GO IN HERE.
    % What is obligatory, what is optional?
    
\begin{code}
<form id="mturk_form" method="POST" action="https://www.mturk.com/mturk/externalSubmit">
    <!-- some MTurk specific information -->
    <textArea id="assignmentId" name="assignmentId" ></textArea><br />
    <textArea id="practiceResp" name="practiceResp" ></textArea><br />
    <input type="hidden" id="userAgent" name="userAgent" />
                
    <!-- hidden fields for the URL parameters that we want to be stored. -->            
    <input type="hidden" id="label" name="label" />
    <input type="hidden" id="condition" name="condition" />
    <input type="hidden" id="reverse" name="reverse" />
    <input type="hidden" id="reverse2" name="reverse2" />
    <input type="hidden" id="list_num" name="list_num" />
    <input id="submitButton" type="submit" name="Submit" value="Submit" />
</form>
\end{code}

    \item A link to whatever Javascript code you're using. For example, the example experiment in this tutorial uses js-adapt, a collection 
    of code originally written by Dave Kleinschmidt, and then extended by Linda Liu, Zach Burchill, Wednesday Bushong, and Xin Xie. That code 
    provides object definitions for experiments, different types of blocks, stimulus lists, etc. These definitions can then be evokes to 
    create the specific experiment, like the last line in the following example, in which experiment-A.js would contain the code for the 
    experiment (code that references code in the other .js files).
    
\begin{code}
<!-- general javascript, incl. general definitions of objects 
     for experiment, block, and stimulus lists  //-->
<script src="js-adapt/jquery-1.10.1.min.js" type="text/javascript" ></script>
<script src="js-adapt/modernizr.min.js" type="text/javascript"></script>
<script src="js-adapt/stimuli.js" type="text/javascript" ></script>
<script src="js-adapt/labelingBlock.js" type="text/javascript" ></script>
<script src="js-adapt/exposureBlock.js" type="text/javascript" ></script>
<script src="js-adapt/experimentControl2.js" type="text/javascript" ></script>
<script src="js-adapt/soundcheckBlock.js" type="text/javascript"></script>
<script src="js-adapt/mturk_helpers.js" type="text/javascript"></script>
<script src="js-adapt/progressBar.js" type="text/javascript"></script>
<script src="js-adapt/logreg.js"  type="text/javascript"></script>
<script src="js-adapt/utilities.js" type="text/javascript"></script>
<script src="get_stim.js" type="text/javascript"></script>

<!-- Here is where your experiment javascript file is specified -->
<script src="experiment-A.js" type="text/javascript"></script>
\end{code}

    \item Any HTML objects references in the Javascript code. This includes, for example, various <div>s that the Javascript code will
    fill with content at different points during the experiment. % ASK WEDNESDAY WHETHER THIS IS CORRECT.
\end{itemize}



\section{Example experiments using {\em JSEXP}}

\begin{itemize}
    \item The code for the unsupervised and supervised learning experiment in Kleinschmidt et. al (2015) \footnote{\href{https://mindmodeling.org/cogsci2015/papers/0200/paper0200.pdf}{https://mindmodeling.org/cogsci2015/papers/0200/paper0200.pdf}} can be cloned from \url{https://bitbucket.org/hlplab/nrtmodule/src/master/}. If you'd like to get a sense of a full version of the experiment, you can test it out at \url{https://www.hlp.rochester.edu/mturk/mtadapt/sup-unsup/}.
    \item The code for a web-based version of the priming experiment from Xie \& Myers (2017) can be found at \url{https://github.com/xinxie-cogsci/Online-experiments}. If you'd like to get a sense of a full version of the experiment, you can test it out at, for example, \url{https://www.hlp.rochester.edu/mturk/xxie/io_perception/exp1/exp1_transcription.html?condition=experimental&speaker=M15&order=2&visual=1&block=2} (for information on the URL parameters, see the readme.md file in the github repository).
    \item And here is a demo of a whole number of different paradigms available on request: \url{https://www.hlp.rochester.edu/mturk/lliu/demos/}.
\end{itemize}


\end{document}
